\documentclass[twofold, twocolumn]{article}

\usepackage[sc]{mathpazo}
\usepackage[T1]{fontenc}
\linespread{1.05}
\usepackage{microtype}

\usepackage[english]{babel}

\usepackage[hmarginratio=1:1, top=32mm,columnsep=20pt]{geometry}
\usepackage[hang, small,labelfont=bf,up,textfont=it,up]{caption}
\usepackage{booktabs}

\usepackage{mathtools}
\usepackage{amsthm}
\usepackage{abstract}
\renewcommand{\abstractnamefont}{\normalfont\bfseries}
\renewcommand{\abstracttextfont}{\normalfont\small}

\usepackage{lettrine}

\usepackage{titlesec}
\renewcommand\thesection{\Roman{section}}
\renewcommand\thesubsection{\roman{subsection}}
\titleformat{\section}[block]{\large\scshape\centering}{\thesection.}{1em}{}
\titleformat{\subsection}[block]{\large}{\thesubsection.}{1em}{}

\usepackage{fancyhdr}
\pagestyle{fancy}
\fancyhead{}
\fancyfoot{}
\fancyhead[C]{Pranshu Gaba $\bullet$ Summer Project 2017 $\bullet$ Sr. No. 13718}
\fancyfoot[RO, LE]{\thepage}


\usepackage{titling}
\usepackage{hyperref}

\usepackage{mathtools}

\setlength{\droptitle}{-6\baselineskip}
\pretitle{\begin{center}\huge\bfseries}
\posttitle{\end{center}}


\newcommand*\conj[1]{\overline{#1}}
\newcommand*\adj[1]{#1^*}
\newcommand*\norm[1]{\left \Vert #1 \right\Vert}
\newcommand*\abs[1]{\left \vert #1 \right\vert}


\theoremstyle{plain}
\newtheorem*{theorem}{Theorem}
\newtheorem*{corollary}{Corollary}
\newtheorem*{lemma}{Lemma}

\theoremstyle{definition}
\newtheorem*{definition}{Definition}



\author{%
\textsc{Pranshu Gaba} \thanks{:)} \\[1ex]
\normalsize Indian Institute of Science, Bangalore \\
\normalsize \href{mailto:pranshu@ug.iisc.in}{pranshu@ug.iisc.in}}
\title{Hermitian Forms and Zeros of a Polynomial}
\date{\today}

\renewcommand{\maketitlehookd}{%

\begin{abstract}
We looked at the general properties of Hermitian (self-adjoint) matrices, and used the Schur-Cohn theorem to find the number of roots of a polynomial lying within and without the unit circle. 
\end{abstract}
}


\begin{document}
\maketitle

\section{Introduction}

\lettrine[nindent=0em,lines=2]{I}n this paper we see the properties of Hermitian matrices, which are very interesting, as well as useful. We also see and prove the Schur-Cohn theorem to find the number of roots of a polynomial lying within the unit circle. 

There are many ways to locate the roots of a polynomial. The Schur-Cohn theorem shows a surprising connection between linear algebra and roots of a polynomial. It will be used to find out how many roots of the polynomial lie inside and outside the unit circle.


\section{Hermitian Matrices}

The adjoint of a matrix \(A \in \mathbb{C}_n\) is the matrix obtained by taking its transpose, followed by taking the complex conjugate of every element. The adjoint of matrix \(A\) is denoted by \(\adj{A}\). If the \(ij^{\text{th}} \) of \(A\) is \(a_{ij}\), then the \(ij^{\text{th}}\) entry of \(\adj{A}\) is \(\conj{a_{ji}}\). We see that \(\adj{A}\) is a linear transformation. The adjoint satisfies \(\langle Ax, y \rangle = \langle x, \adj{A}y \rangle\). 

Hermitian matrices (also known as self-adjoint matrices) are matrices that satisfy \(A = \adj{A}\). All the eigenvalues of a Hermitian matrix are real. 

\begin{definition} Any matrix \(B \in \mathbb{M}_n\) that satisfies \(\langle Bx, x\rangle \ge 0\) for all \(x \in \mathbb{C}^n\) is called a positive semidefinite matrix. \end{definition}

\begin{corollary}All the eigenvalues of positive semidefinite matrix are non-negative. \end{corollary}




\(\adj{A} A\) is always positive semidefinite. 

Hermitian matrices can be diagonalized. For every Hermitian matrix \(A\), there exists a diagonal matrix \(\Lambda\) such that \(A = \adj{U}  \Lambda U\). Here \(U\) is some unitary matrix. 



\begin{lemma} If \(A \in \mathbb{M}_n (\mathbb{C})\) and \(\langle Ax, x \rangle \in \mathbb{R}\) for every \(x\), then \(A = \adj{A}\). \end{lemma}
\begin{proof} Let \(\alpha \in \mathbb{C}\) and \(h, g \in \mathbb{C}^n\). Then `\end{proof}

\begin{corollary}Every positive semidefinite matrix is Hermitian.\end{corollary}
\section{Schur-Cohn Theorem}

Given any polynomial \(p(z) = a_0 z^n + a_1z^{n-1} + \cdots + a_n\) with complex coefficients, we are interested in finding how many of its roots lie within the unit circle and how many roots lie outside. Without loss of generality, let \(a_0 = 1\) as it does not change the roots of the polynomial. 

Suppose \(p\) has roots \(\alpha_i\). Then \(p(z) = (z - \alpha_1) (z - \alpha_2) \cdots (z - \alpha_n)\). 


Let \(S\) be the \(n \times n\) square matrix \( \begin{bmatrix} 
0 & 1 & 0 & \ldots & 0 \\
0 & 0 & 1 & \ldots & 0 \\
\vdots & \vdots & \vdots & & \vdots \\
0 & 0 & 0 &\ldots & 1 \\
0 & 0 & 0 & \ldots & 0 \\ 
\end{bmatrix}\).  Note that \(S\) is a nilpotent matrix of order \(n\), i.e. \(S^n\) is a zero matrix. Then \(p(S) \) is \( \begin{bmatrix} 

% replace zeroes with dots
a_n & a_{n-1} & \ddots & \ddots & a_1 \\
0 & a_n & a_{n-1} & \ddots & \ddots \\
0 & 0 & a_n & \ddots & \ddots \\
0 & 0 & 0 &\ddots & a_{n-1} \\
0 & 0 & 0 & 0 & a_n \\ 
\end{bmatrix}\). 


This can be factorized as \(p(S) = (S - \alpha_1I) (S - \alpha_2 I) \cdots (S - \alpha_n I)\). Let \(B_j = S - \alpha_jI\).

Next, define \(q\) as the polynomial \(\conj{a_n}z^n + \conj{a_{n-1}}z^{n-1} + \cdots + \conj{a_0}\). Note that its roots are \(\frac {1}{\conj{\alpha_i}}\). We get \(q(z) = (1 - \conj{\alpha_1}z) (1 - \conj{\alpha_2}z) \cdots (1 - \conj{\alpha_nz})\). Also, \(q(S) = (I - \conj{\alpha_1}S) (I - \conj{\alpha_2}S) \cdots (I - \conj{\alpha_n}S)\). Let \(C_j= I -  \conj{\alpha_j} S\).
%Explain how roots of q are obtained.


Let \(H\) be equal to \(\lVert q(S) x \rVert^2 - \lVert p(S) x \rVert^2\)

%derive this expression of H
\(H\) can also be written as \(\langle (\adj{q(S)} q(S) - \adj{p(S)} p(S))x, x\rangle\).

We can now state the Schur-Cohn theorem:

\begin{theorem}The polynomial \(p\), it will have \(k\) roots inside the circle, and \(n-k\) roots outside the circle iff \(k\) eigenvalues of \(H\) are positive and \(n-k\) are negative. \end{theorem}

\section{Proof}
%The proof is trivial and is left as an exercise to the reader. 

Let's write \(q(S)\) and \(p(S)\) as a product of the linear terms. \(\adj{q(S)} q(S) - \adj{p(S)} p(S) \\= \adj{(C_1C_2C_3 \ldots C_n)}(C_1C_2C_3\ldots C_n) - \adj{(B_1B_2B_3\ldots B_n)} (B_1B_2B_3\ldots B_n)\)

Let's look at \(\adj{C_1} C_1 - \adj{B_1} B_1\) first. Substituting the values of \(C_1\) and \(B_1\), we get 

\begin{equation*}
\begin{split}
& \phantom{=}    \adj{C_1}C_1 - \adj{B_1} B_1 \\
 & = \adj{(I - \conj{\alpha_1}S)} (I - \conj{\alpha_1}S) - \adj{(S - \alpha_1 I)} (S - \alpha_1 I) \\
& = (I - \alpha_1\adj{S}) (I - \conj{\alpha_1}S) - (\adj{S} - \conj{\alpha_1} I) (S - \alpha_1 I) \\
 & = (I - \alpha_1\adj{S} - \conj{\alpha_1}S + \abs{\alpha_1}^2 \adj{S} S) - (\adj{S} S - \alpha_1 \adj{S} - \conj{\alpha_1} S + \abs{\alpha_1}^2I)\\
& = I - \abs{\alpha_1}^2 I - \adj{S} S + \abs{\alpha_1}^2 \adj{S} S \\
& = (1 - \abs{\alpha_1}^2) (I - \adj{S} S)
\end{split}
\end{equation*}

Note that \(I - \adj{S} S\) is a posititve definite matrix. If \(\abs{\alpha} < 1\), then the root of the linear polynomial lies within the unit circle. Also note that \(H\) has one negative eigenvalue. Similarly, if \(\abs{\alpha} > 1\), then the root of the linear polynomial lies outside the unit circle, and the eigenvalue of \(H\) is positive. This shows that the Schur-Cohn theorem is true for \(n = 1\). We will now extend the proof for all \(n\). 
\section{Extensions}

\section{Conclusion}

\end{document}