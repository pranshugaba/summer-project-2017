\documentclass[twofold, twocolumn]{article}

\usepackage[sc]{mathpazo}
\usepackage[T1]{fontenc}
\linespread{1.05}
\usepackage{microtype}

\usepackage[english]{babel}

\usepackage[hmarginratio=1:1, top=32mm,columnsep=20pt]{geometry}
\usepackage[hang, small,labelfont=bf,up,textfont=it,up]{caption}
\usepackage{booktabs}

\usepackage{amsthm}
\usepackage{abstract}
\renewcommand{\abstractnamefont}{\normalfont\bfseries}
\renewcommand{\abstracttextfont}{\normalfont\small}

\usepackage{lettrine}

\usepackage{titlesec}
\renewcommand\thesection{\Roman{section}}
\renewcommand\thesubsection{\roman{subsection}}
\titleformat{\section}[block]{\large\scshape\centering}{\thesection.}{1em}{}
\titleformat{\subsection}[block]{\large}{\thesubsection.}{1em}{}

\usepackage{fancyhdr}
\pagestyle{fancy}
\fancyhead{}
\fancyfoot{}
\fancyhead[C]{Pranshu Gaba $\bullet$ Summer Project 2017 $\bullet$ Sr. No. 13718}
\fancyfoot[RO, LE]{\thepage}


\usepackage{titling}
\usepackage{hyperref}

\usepackage{mathtools}

\setlength{\droptitle}{-6\baselineskip}
\pretitle{\begin{center}\huge\bfseries}
\posttitle{\end{center}}


\newcommand*\conj[1]{\overline{#1}}

\newtheorem*{theorem}{Theorem}
\newtheorem*{definition}{Definition}
\newtheorem*{corollary}{Corollary}



\author{%
\textsc{Pranshu Gaba} \thanks{:)} \\[1ex]
\normalsize Indian Institute of Science, Bangalore \\
\normalsize \href{mailto:pranshu@ug.iisc.in}{pranshu@ug.iisc.in}}
\title{Hermitian Forms and Zeros of a Polynomial}
\date{\today}

\renewcommand{\maketitlehookd}{%

\begin{abstract}
We looked at the general properties of Hermitian (self-adjoint) matrices, and used the Schur-Cohn theorem to find the number of roots of a polynomial lying within and without the unit circle. 
\end{abstract}
}


\begin{document}
\maketitle

\section{Introduction}

\lettrine[nindent=0em,lines=2]{I}n this paper we see the properties of Hermitian matrices, which are very interesting, as well as useful. We also see and prove the Schur-Cohn theorem to find the number of roots of a polynomial lying within the unit circle. 

There are many ways to locate the roots of a polynomial. Using the Schur-Cohn theorem gives a nice estimate on how many roots lie inside the unit circle.


\section{Hermitian Matrices}

The adjoint of a matrix is its conjugate transpose. The \(ij\)th entry of \(A^*\) is \(\conj{a_{ji}}\)

Hermitian matrices (also known as self-adjoint matrices) are matrices that satisfy \(A^* = A\). All the eigenvalues of a Hermitian matrix are real. 

\begin{definition} Any matrix \(B \in \mathbb{M}_n\) that satisfies \(\langle Bx, x\rangle \ge 0\) for all \(x \in \mathbb{C}^n\) is called a positive semidefinite matrix. \end{definition}

\begin{corollary}All the eigenvalues of positive semidefinite matrix are non-negative. \end{corollary}

\begin{corollary}Every positive semidefinite matrix is Hermitian.\end{corollary}


Hermitian matrices can be diagonalized. For every Hermitian matrix A, there exists a diagonal matrix \(\Lambda\) such that \(A = U^* \Lambda U\). Here \(U\) is some unitary matrix. 

\section{Schur-Cohn Theorem}

Given a polynomial \(p(z) = a_0 z^n + a_1z^{n-1} + \cdots + a_n\). Suppose \(p\) has roots \(\alpha_i\). Then \(p(z) = (z - \alpha_1) (z - \alpha_2) \cdots (z - \alpha_n)\). 

Without loss of generality, let \(a_0 = 1\) as it does not change the roots of the polynomial. 

Let \(S\) be the \(n \times n\) square matrix \( \begin{bmatrix} 
0 & 1 & 0 & \ldots & 0 \\
0 & 0 & 1 & \ldots & 0 \\
\vdots & \vdots & \vdots & & \vdots \\
0 & 0 & 0 &\ldots & 0 \\
0 & 0 & 0 & \ldots & 0 \\ 
\end{bmatrix}\).  Note that it is nilpotent of order \(n\), i.e. \(S^n\) is a zero matrix. Then \(p(S) \) is \( \begin{bmatrix} 
a_n & a_{n-1} & \ddots & \ddots & a_1 \\
0 & a_n & a_{n-1} & \ddots & \ddots \\
0 & 0 & a_n & \ddots & \ddots \\
0 & 0 & 0 &\ddots & a_{n-1} \\
0 & 0 & 0 & 0 & a_n \\ 
\end{bmatrix}\). 


This can be factorized as \(p(S) = (S - \alpha_1I) (S - \alpha_2 I) \cdots (S - \alpha_n I)\). 


Then define \(q\) as the polynomial with roots \(\frac {1}{\conj{\alpha_i}}\). We get \(q(z) = (1 - \conj{\alpha_1}z) (1 - \conj{\alpha_2}z) \cdots (1 - \conj{\alpha_nz})\)


Let \(H\) be equal to \(\lVert q(S) x \rVert^2 - \lVert p(S) x \rVert^2\)


\begin{theorem}The polynomial \(p\), it will have \(k\) roots inside the circle, and \(n-k\) roots outside the circle iff \(k\) eigenvalues of \(H\) are positive and \(n-k\) are negative. \end{theorem}

\section{Proof}
%The proof is trivial and is left as an exercise to the reader. 

\(q(S)^* q(S) - p(S)^* p(S) \\
= (C_1C_2C_3 \ldots C_n)^*(C_1C_2C_3\ldots C_n) - (B_1B_2B_3\ldots B_n)^*(B_1B_2B_3\ldots B_n)\)

\section{Extensions}

\section{Conclusion}

\end{document}
