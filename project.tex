

\documentclass[twofold]{article}

%\usepackage[sc]{mathpazo}
\usepackage{lmodern}
\usepackage{amssymb}
\usepackage[T1]{fontenc}
\linespread{1.05}
\usepackage{microtype}


\usepackage[english]{babel}

\usepackage[hmarginratio=1:1, top=32mm,columnsep=20pt]{geometry}
\usepackage[hang, small,labelfont=bf,up,textfont=it,up]{caption}
\usepackage{booktabs}

\usepackage{mathtools}
\usepackage{amsthm}
\usepackage{abstract}
\renewcommand{\abstractnamefont}{\normalfont\bfseries}
\renewcommand{\abstracttextfont}{\normalfont\small}

\usepackage{lettrine}

%\usepackage{titlesec}
%\renewcommand\thesection{\Roman{section}}
%\renewcommand\thesubsection{\roman{subsection}}
%\titleformat{\section}[block]{\large\scshape\centering}{\thesection.}{1em}{}
%\titleformat{\subsection}[block]{\large}{\thesubsection.}{1em}{}

\usepackage{fancyhdr}
\pagestyle{fancy}
\fancyhead{}
\fancyfoot{}
\fancyhead[C]{Pranshu Gaba $\bullet$ Summer Project 2017 $\bullet$ Sr. No. 13718}
\fancyfoot[RO, LE]{\thepage}


\usepackage{titling}
\usepackage{hyperref}

\usepackage{mathtools}

%\setlength{\droptitle}{-6\baselineskip}
\pretitle{\begin{center}\huge\bfseries}
\posttitle{\end{center}}


\newcommand*\conj[1]{\overline{#1}}
\newcommand*\adj[1]{#1^*}
\newcommand*\norm[1]{\left \Vert #1 \right\Vert}
\newcommand*\abs[1]{\left \vert #1 \right\vert}
\DeclareMathOperator{\Tr}{Tr}
\DeclarePairedDelimiterX{\inp}[2]{\langle}{\rangle}{#1, #2}


\theoremstyle{plain}
\newtheorem*{theorem}{Theorem}
\newtheorem*{corollary}{Corollary}
\newtheorem*{lemma}{Lemma}

\theoremstyle{definition}
\newtheorem*{definition}{Definition}



\author{%
\textsc{Pranshu Gaba} \thanks{:)} \\[1ex]
\normalsize Indian Institute of Science, Bangalore \\
\normalsize \href{mailto:gabapranshu@iisc.ac.in}{gabapranshu@ug.iisc.in}}
\title{Hermitian Forms and Zeros of a Polynomial}
\date{\today}

\renewcommand{\maketitlehookd}{%

\begin{abstract}
We looked at the general properties of Hermitian (self-adjoint) matrices, and used the Schur-Cohn theorem to find the number of roots of a polynomial lying within and without the unit circle. 
\end{abstract}
}


\begin{document}
\maketitle

\section{Introduction}

%\lettrine[nindent=0em,lines=2]{I}



In this paper we see the properties of Hermitian matrices, which are very interesting, as well as useful. We also see and prove the Schur-Cohn theorem to find the number of roots of a polynomial lying within the unit circle. 

There are many ways to locate the roots of a polynomial. The Schur-Cohn theorem shows a surprising connection between linear algebra and roots of a polynomial. It will be used to find out how many roots of the polynomial lie inside and outside the unit circle.


\section{Definitions}


\subsection{Inner product}

A binary operator \(\inp{\cdot}{\cdot} \colon \mathbb{C}^n \times \mathbb{C}^n \to \mathbb{C}\). 

\begin{itemize}
\item It is linear in the first term. 

\(\inp{ax}{y} = a \inp{x}{y}\)
\(\inp{x + y}{z} = \inp{x}{z} + \inp{y}{z}\)

\item It is conjugate when commutated

\(\inp{x}{y} = \conj{\inp{y}{x}}\)

\item Semi-positive definite

\(\inp{x}{x} \ge 0\) for all \(x\).

Equality is achieved if and only if \(x = 0\).
\end{itemize}


\subsection{Adjoint}

The adjoint of a matrix \(A \in \mathbb{C}_n\), denoted by \(\adj{A}\), is the matrix that  satisfies \(\inp{\adj{A}x}{y} = \inp{x}{Ay}\). 

The adjoint can be obtained by taking its transpose, followed by taking the complex conjugate of every element. If the \(ij^{\text{th}} \) of \(A\) is \(a_{ij}\), then the \(ij^{\text{th}}\) entry of \(\adj{A}\) is \(\conj{a_{ji}}\). Note that \(\adj{A}\) is a linear transformation.

\(a_{ij}\) is defined as \(\inp{Ae_j}{e_i}\).

\subsection{Positive Definite}

\begin{definition} Any matrix \(B \in \mathbb{M}_n\) that satisfies \(\langle Bx, x\rangle \ge 0\) for all \(x \in \mathbb{C}^n\) is called a positive semidefinite matrix. \end{definition}

\begin{theorem}All the eigenvalues of positive semidefinite matrix are non-negative. \end{theorem}
\begin{proof} Left to the reader. \end{proof}


\subsection{Hermitian Matrices}
Hermitian matrices (also known as self-adjoint matrices) are matrices that satisfy \(A = \adj{A}\). 

\begin{theorem} All the eigenvalues of a Hermitian matrix are real. \end{theorem}

\begin{proof}
Let \(v\) be an eigenvector of a Hermitian matrix, \(A\), and let \(\lambda\) be the corresponding eigenvalue. Then \(Av = \lambda v\). 

\(\inp{Av}{v} = \inp{\lambda v}{v} = \lambda\inp{v}{v}\). Also \(\inp{Av}{v} = \inp{v}{\adj{A}v} = \inp{v}{Av} = \inp{v}{\lambda v} = \conj{\lambda} \inp{v}{v}\)

This means \(\lambda \inp{v}{v} = \conj{\lambda} \inp{v}{v}\) for any \(v\). Since \(\lambda\) is the same conjugate as its complex conjugate, it implies \(\lambda\) is real. 
 \end{proof}


The converse of this is also true. 
\begin{theorem} If \(A \in \mathbb{M}_n (\mathbb{C})\) and \(\inp{Ax}{x} \in \mathbb{R}\) for every \(x\), then \(A = \adj{A}\). \end{theorem}

\begin{proof} Let \(\alpha \in \mathbb{C}\) and \(h, g \in \mathbb{C}^n\). Then
\(\inp{A(h + \alpha g)}{h + \alpha g} = \inp{Ah}{h} + \alpha \inp{Ag}{h} + \conj{\alpha} \inp{Ah}{g} + \abs{\alpha}^2 \inp{Ag}{g} \)

So \(\alpha \inp{Ag}{h} + \conj{\alpha} \inp{Ah}{g} = \conj{\alpha} \inp{h}{Ag} + \alpha \inp{g}{Ah} \)

When \(\alpha = 1\), \(\inp{Ag}{h} + \inp{Ah}{g} = \inp{h}{Ag} + \inp{g}{Ah}\)\

When \(\alpha = i\), \(i \inp{Ag}{h} - i \inp{Ah}{g} = -i \inp{h}{Ag} + i \inp{g}{Ah}\)

\(2i \inp{Ag}{h} = 2i \inp{g}{Ah}\) ot \(\inp{Ag}{h} = \inp{g}{Ah} = \inp{\adj{A}g}{h}\)

\(Ag = \adj{A}g\) for all \(g\), therefore \(A = \adj{A}\). \(A\) is Hermitian.
 \end{proof}

\begin{corollary}Every positive semidefinite matrix is Hermitian.\end{corollary}
\begin{proof} Left to the reader. \end{proof}

The converse is also true.
\begin{theorem} \(\adj{A} A\) is always positive semidefinite. \end{theorem}

\begin{proof} \(\inp{\adj{A} A x}{ x} = \inp{Ax}{Ax} = \norm{Ax}^2 \ge 0\)\end{proof}




\subsection{Unitary Matrices}

A square matrix \(U\) is a unitary matrix if \(\adj{U} U = I\). The determinant of a unitary matrix is 1. It preserves inner product, \(\inp{Ux}{Uy} = \inp{x}{y}\).

\subsection{Norm of a matrix}

\subsubsection{Operator norm}
Given \(A \in \mathbb{M}_n\), define \(\norm{A} =\displaystyle \sup _{x \neq 0} \frac{\norm{Ax}}{\norm{x}} = \sup_{\norm{x} = 1} \norm{Ax} \) to be the operator norm of \(A\). The triangle inequality \(\norm{A + B} \leq \norm{A} + \norm{B}\) is satisfied. 

\subsubsection{Hilbert-Schmidt norm}

The Hilbert-Schmidt norm of matrix \(A\), is defined as the square root of sum of squares of all entries in \(A\). 

 \[\norm{A}_2 = \left( \sum_{i, j} \abs{a_{ij}}^2 \right) ^{1/2}\]


The operator norm is always less than or equal to the Hilbert-Schmidt norm.

\subsection{Trace}
The trace of a matrix is the sum of the diagonal elements of the matrix. 

\(\Tr(A) = \sum_{i=1} ^n \inp{Ae_i}{ e_i}\)

\begin{theorem}The trace of \(\adj{A}A\) is equal to the Hilbert-Schmidt norm of \(A\). \(\Tr( \adj{A} A ) = \norm{A}_2\)\end{theorem}

\subsection{Diagonalization}

Hermitian matrices can be diagonalized. For every Hermitian matrix \(A\), there exists a diagonal matrix \(\Lambda\) such that \(A = \adj{U}  \Lambda U\). Here \(U\) is some unitary matrix. 


\subsection{Projectors}
A matrix \(P\) is a projector if \(P^2 = P\) and \(\adj{P} = P\)



\section{Schur-Cohn Theorem}

Given any polynomial \(p(z) = a_0 z^n + a_1z^{n-1} + \cdots + a_n\) with complex coefficients, we are interested in finding how many of its roots lie within the unit circle and how many roots lie outside. Without loss of generality, let \(a_0 = 1\) as it does not change the roots of the polynomial. 

Suppose \(p\) has roots \(\alpha_i\). Then \(p(z) = (z - \alpha_1) (z - \alpha_2) \cdots (z - \alpha_n)\). 

Let \(S\) be the \(n \times n\) square matrix \( \begin{bmatrix} 
0 & 1 & 0 & \ldots & 0 \\
0 & 0 & 1 & \ldots & 0 \\
\vdots & \vdots & \vdots &\ddots & \vdots \\
0 & 0 & 0 &\ldots & 1 \\
0 & 0 & 0 & \ldots & 0 \\ 
\end{bmatrix}\). 

 Note that \(S\) is a nilpotent matrix of order \(n\), i.e. \(S^n\) is a zero matrix.



 Then \(p(S) \) is \( \begin{bmatrix} 

% replace zeroes with dots
a_n & a_{n-1} & \ddots & \ddots & a_1 \\
0 & a_n & a_{n-1} & \ddots & \ddots \\
0 & 0 & a_n & \ddots & \ddots \\
0 & 0 & 0 &\ddots & a_{n-1} \\
0 & 0 & 0 & 0 & a_n \\ 
\end{bmatrix}\). 


This can be factorized as \(p(S) = (S - \alpha_1I) (S - \alpha_2 I) \cdots (S - \alpha_n I)\). Let \(B_j = S - \alpha_jI\).

Next, define \(q\) as the polynomial \(\conj{a_n}z^n + \conj{a_{n-1}}z^{n-1} + \cdots + \conj{a_0}\). Note that its roots are \(\frac {1}{\conj{\alpha_i}}\). We get \(q(z) = (1 - \conj{\alpha_1}z) (1 - \conj{\alpha_2}z) \cdots (1 - \conj{\alpha_nz})\). Also, \(q(S) = (I - \conj{\alpha_1}S) (I - \conj{\alpha_2}S) \cdots (I - \conj{\alpha_n}S)\). Let \(C_j= I -  \conj{\alpha_j} S\).
%Explain how roots of q are obtained.


Let \(H\) be equal to \(\norm{ q(S) x }^2 - \norm{ p(S) x}^2\)

%derive this expression of H
\(H\) can also be written as \(\inp{(\adj{q(S)} q(S) - \adj{p(S)} p(S))x}{ x}\).

We can now state the Schur-Cohn theorem:

\begin{theorem}The polynomial \(p\), it will have \(k\) roots inside the circle, and \(n-k\) roots outside the circle iff \(k\) eigenvalues of \(H\) are positive and \(n-k\) are negative. \end{theorem}

\section{Proof}
%The proof is trivial and is left as an exercise to the reader. 

We will first prove the Schur-Cohn theorem for \(n =1\), that is for linear polynomials. It will then be extended to polynomials of higher degrees with the help of the Spectral theorem and the Courant-Fischer theorem. 


\subsection{Linear Polynomial}


Let's write \(q(S)\) and \(p(S)\) as a product of the linear terms. \(\adj{q(S)} q(S) - \adj{p(S)} p(S) \\= \adj{(C_1C_2C_3 \ldots C_n)}(C_1C_2C_3\ldots C_n) - \adj{(B_1B_2B_3\ldots B_n)} (B_1B_2B_3\ldots B_n)\)

Let's look at \(\adj{C_1} C_1 - \adj{B_1} B_1\) first. Substituting the values of \(C_1\) and \(B_1\), we get 

\begin{equation*}
\begin{split}
& \phantom{=}    \adj{C_1}C_1 - \adj{B_1} B_1 \\
 & = \adj{(I - \conj{\alpha_1}S)} (I - \conj{\alpha_1}S) - \adj{(S - \alpha_1 I)} (S - \alpha_1 I) \\
& = (I - \alpha_1\adj{S}) (I - \conj{\alpha_1}S) - (\adj{S} - \conj{\alpha_1} I) (S - \alpha_1 I) \\
 & = (I - \alpha_1\adj{S} - \conj{\alpha_1}S + \abs{\alpha_1}^2 \adj{S} S) - (\adj{S} S - \alpha_1 \adj{S} - \conj{\alpha_1} S + \abs{\alpha_1}^2I)\\
& = I - \abs{\alpha_1}^2 I - \adj{S} S + \abs{\alpha_1}^2 \adj{S} S \\
& = (1 - \abs{\alpha_1}^2) (I - \adj{S} S)
\end{split}
\end{equation*}

Note that \(I - \adj{S} S\) is a posititve definite matrix. If \(\abs{\alpha} < 1\), then the root of the linear polynomial lies within the unit circle. Also note that \(H\) has one negative eigenvalue. Similarly, if \(\abs{\alpha} > 1\), then the root of the linear polynomial lies outside the unit circle, and the eigenvalue of \(H\) is positive. This shows that the Schur-Cohn theorem is true for \(n = 1\). We will now extend the proof for all \(n\). 

\subsection{Spectral theorem}

\begin{theorem} Let \(A \in \mathbb{M}_n\) be a Hermitian matrix with eigenvalues \(\lambda_1 \le \lambda_2 \le \lambda_3 \le \cdots \le \lambda_n\). Then \(A\) can be written as \(U \Lambda \adj{U} \), where \(U\) is a unitary matrix, and \(\Lambda\) is a diagonal matrix with real entries. \end{theorem}

\begin{proof} Left to the reader. \end{proof}

\subsection{Courant-Fischer theorem}

\begin{theorem} Let \(A \in \mathbb{M}_n\) be a Hermitian matrix with eigenvalues \(\lambda_1 \le \lambda_2 \le \lambda_3 \le \cdots \le \lambda_n\). Then 

  \[ \lambda_k = \min_{\omega_1, \ldots , \omega_{n-k} \in \mathbb{C}^n} \max_{\substack{x \neq 0, x\in \mathbb{C}^n \\ x \perp \omega_1, \ldots , \omega_{n-k}}} \frac{\inp{Ax}{x}}{\inp{x}{x}}\]

 \end{theorem}




\begin{proof} If \(x \neq 0\), then \(\frac{\inp{Ax}{x}}{\inp{x}{x}} = \frac{\inp{U \Lambda \adj{U} x}{x}}{\inp{\adj{U} x}{ \adj{U} x}} \\
 = \frac{\inp{\Lambda \adj{U} x}{ \adj{U} x}}{\inp{\adj{U}x}{\adj{U}x}}\). and \( \{ \adj{U} x : x \neq 0\}  = \{ x \in \mathbb{C}^n : x \neq 0 \}\) 



Thus if \(\omega_1, \ldots , \omega_{n-k}\) are given, then 

\[ \sup_{\substack{x \neq 0 \\ x \perp \omega_1, \ldots, \ \omega_{n-k}}} \frac{\inp{Ax}{x}}{\inp{x}{x}} = \sup_{\substack{y \neq 0 \\ y \perp \adj{U}\omega_1, \ldots ,\ \adj{U} \omega_{n-k}}} \frac{\inp{\Lambda y}{y}}{\inp{y}{y}}\]

\(x \perp \omega\) if and only if \(y \perp \adj{U} \omega\). 

\[ = \sup_{\substack{\inp{y}{y} = 1 \\ y \perp \adj{U} \omega_1, \ \ldots , \ \adj{U}\omega_{n-k}}} \sum_{i = 1}^n \lambda_i \abs{y_i}\]

\[ \ge \sup_{\substack{\inp{y}{y} = 1 \\ y \perp \adj{U} \omega_1, \ \ldots , \ \adj{U}\omega_{n-k} \\ y_1 = y_2 = \cdots = y_k-1 = 0 }} \sum_{i = 1}^n \lambda_i \abs{x_i}^2\]

\[ = \sup_{\substack{\inp{y}{y} = 1 \\ y \perp \adj{U} \omega_1, \ \ldots , \ \adj{U}\omega_{n-k} \\ y_1 = y_2 = \cdots = y_k-1 = 0 }} \sum_{i = k}^n \lambda_i \abs{y_i}^2\]

\[ \ge \lambda_k \]


Let \(\omega_1 = x_n , \ldots , \ \omega_{n-k} = x_{k + 1}\)

If \(x\perp \omega_i\), as above, then \(x = \sum_{i = 1} ^k c_i x_i\). 

\(\inp{Ax}{x} = \inp{A \sum_{i = 1} ^ n c_i X_i}{ \sum_{i = 1} ^n c_i x_i}\)

\( = \inp{\sum_{i = 1} ^n c_i \lambda_i x_i}{\sum_{i = 1} ^n c_i x_i}\)

\(= \sum_{i = 1} ^ k \lambda_i \abs{c_i}^2\) 

\( \le \lambda_k \sum_{i = 1} ^{k} \abs{c_i}^2\)


\end{proof}

%\section{Extensions}


%\subsection{Arbitrary radius}

%\subsection{Limitations}

\section{Conclusion}
Thank You!

\section{Acknowledgements}

\end{document}