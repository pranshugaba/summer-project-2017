\documentclass[twofold, twocolumn]{article}

\usepackage[sc]{mathpazo}
\usepackage[T1]{fontenc}
\linespread{1.05}
\usepackage{microtype}

\usepackage[english]{babel}

\usepackage[hmarginratio=1:1, top=32mm,columnsep=20pt]{geometry}
\usepackage[hang, small,labelfont=bf,up,textfont=it,up]{caption}
\usepackage{booktabs}

%\usepackage{mathtools}
\usepackage{abstract}
\renewcommand{\abstractnamefont}{\normalfont\bfseries}
\renewcommand{\abstracttextfont}{\normalfont\small}

\usepackage{lettrine}

\usepackage{titlesec}
\renewcommand\thesection{\Roman{section}}
\renewcommand\thesubsection{\roman{subsection}}
\titleformat{\section}[block]{\large\scshape\centering}{\thesection.}{1em}{}
\titleformat{\subsection}[block]{\large}{\thesubsection.}{1em}{}

\usepackage{fancyhdr}
\pagestyle{fancy}
\fancyhead{}
\fancyfoot{}
\fancyhead[C]{Running title $\bullet$ May 2016 $\bullet$ Vol. XXI, No. 1}
\fancyfoot[RO, LE]{\thepage}


\usepackage{titling}
\usepackage{hyperref}

\usepackage{mathtools}

\setlength{\droptitle}{-4\baselineskip}
\pretitle{\begin{center}\huge\bfseries}
\posttitle{\end{center}}


\newcommand*\conj[1]{\bar{#1}}



\author{%
\textsc{Pranshu Gaba} \thanks{:)} \\[1ex]
\normalsize Indian Institute of Science, Bangalore \\
\normalsize \href{mailto:pranshu@ug.iisc.in}{pranshu@ug.iisc.in}}
\title{Hermitian Forms and Zeros of a Polynomial}
\date{\today}

\renewcommand{\maketitlehookd}{%

\begin{abstract}
We looked at the general properties of Hermitian (self-adjoint) matrices, and used the Schur-Conn theorem to find the number of roots of a polynomial lying within the unit circle. 
\end{abstract}
}


\begin{document}
\maketitle

\section{Introduction}

\lettrine[nindent=0em,lines=2]{I}n this project we see the properties of Hermitian matrices, which are very interesting, as well as useful. We also see and prove the Schur-Conn theorem to find the number of roots of a polynomial lying within the unit circle. 

There are many ways to locate the roots of a polynomial. Using the Schur-Conn theorem gives a nice estimate on how many roots lie inside the unit circle.


\section{Hermitian Matrices}
Hermitian matrices (also known as self-adjoint matrices) satisfy \(A^* = A\). 

Hermitian matrices can be diagonalized. For every Hermitian matrix A, there exists a diagonal matrix \(\Lambda\) such that \(A = U^* \Lambda U\). Here \(U\) is some unitary matrix. 

\section{Schur-Conn Theorem}

Given a polynomial \(p(z) = a_0 z^n + a_1z^{n-1} + \cdots + a_n\). Suppose \(p\) has roots \(\alpha_i\). Then \(p(z) = (z - \alpha_1) (z - \alpha_2) \cdots (z - \alpha_n)\). 

Without loss of generality, let \(a_0 = 1\) as it does not change the roots of the polynomial. 

Let \(S\) be the \(n \times n\) square matrix. Note that it is nilpotent of order \(n\), i.e. \(S^n\) is a zero matrix. Then \(p(S) = \text{Insert matrix here}\). 


\(p(S) = (S - \alpha_1I) (S - \alpha_2 I) \cdots (S - \alpha_n I)\). 

Then define \(q\) as the polynomial with roots \(\frac {1}{\conj{\alpha_i}}\). We get \(q(z) = (1 - \conj{\alpha_1}z) (1 - \conj{\alpha_2}z) \cdots (1 - \conj{\alpha_nz})\)


Let \(H\) be equal to \(\lVert q(S) x \rVert^2 - \lVert p(S) x \rVert^2\)


The polynomial \(p\), it will have \(k\) roots inside the circle, and \(n-k\) roots outside the circle iff \(k\) eigenvalues of \(H\) are positive and \(n-k\) are negative. 

\section{Proof}
The proof is trivial and is left as an exercise to the reader. 

\section{Extensions}

\section{Conclusion}

\end{document}
